\documentclass[12pt]{article}
\usepackage{amsmath,amssymb}
\author{Peter Schmidt-Nielsen}
\title{Tausch}
\begin{document}
\maketitle
\begin{abstract}
Tausch is a distributed protocol for constructing bandwidth-efficient distributed broadcast channels.
The protocol provides anonymity for those listening to broadcast channels, without use of any trusted servers.
Assuming $n$ users each transmitting at a constant rate, the naive solution of simply sending all broadcast messages to all users has $\mathcal{O}(n)$ time and bandwidth complexity for each user, and $\mathcal{O}(n^2)$ bandwidth complexity for the network.
Tausch allows many dynamic time/bandwidth tradeoffs, but in the asymptotically optimal configuration, each user has a $\mathcal{O}(n)$ time burden, $\mathcal{O}(\sqrt{n})$ bandwidth burden, and the network as a whole experiences $\mathcal{O}(n^{1.5})$ bandwith complexity.
Secure delegation is possible, allowing a user to experience $\mathcal{O}(1)$ time and bandwidth burdens.
\end{abstract}

\section{Introduction}
Tausch is a protocol (with accompanying implementation) for constructing anonymous broadcast channels.
Each channel can be transmitted to only by a single user, and can be recieved by any number of users.
No party can determine if a given user is recieving a given channel, but can determine the number of channels he or she is recieving.
A secure instantiation of this functionality can be built by simply having every user broadcast every message to every other user.

\end{document}
